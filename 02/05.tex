\subsection{%
  Разбиения множеств. Числа Стирлинга второго рода.%
}

\begin{definition}
  Количество способов разбить множество из \(n\) элементов на \(k\) непустых
  непересекающихся подмножеств (так, чтобы они в объединении давали исходное
  множество) определяется числами Стирлинга второго рода.
\end{definition}

\begin{definition}
  Числа Стирлинга второго рода определяются формулой

  \begin{align*}
    S(n, k) = S(n - 1, k - 1) + k \cdot S(n - 1, k)
  \end{align*}
\end{definition}

Эта формула получается следующим образом: пусть требуется разбить множество
размера \(n\) на \(k\) частей. Есть два варианта действий:
\begin{itemize}
  \item Взять один из элементов в отдельную часть, а оставшиеся \(n - 1\)
  элементов разделить на \(k - 1\) частей.

  \item Разделить \(n - 1\) элемент на \(k\) частей, а оставшийся элемент
  добавить в любую из получившихся \(k\) частей
\end{itemize}
Эти два варианта и определяют два слагаемых в формуле выше.

\begin{remark}
  О крайних случаях

  \begin{itemize}
    \item \(S(n, n) = 1\), причем \(S(0, 0) = 1\)
    
    \item \(S(n, k) = 0\) если \(k \le 0\), т.к. мы не может разделить множество
    на неположительное число частей.

    \item \(S(n, k) = 0\), если \(n < k\), т.к. подмножества должны быть
    непустыми.
  \end{itemize}
\end{remark}

\begin{remark}
  Количество способов разбить множество на произвольное число подмножеств
  (т.е.\(k = 1 \dotsc n\)) называется числами Белла:

  \begin{align*}
    B(n) = \sum_{k = 1}^{n} S(n, k)
  \end{align*}
\end{remark}
