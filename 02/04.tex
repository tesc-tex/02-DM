\subsection{%
  Композиции.%
}

\begin{definition}
  Слабой композицей числа \(n \ge 0\) на \(k\) частей называется решение
  уравнения \(b_{1} + \dotsc + b_{k} = n\) при условии
  \(b_{i} \ge 0, b_{i} \in \ZZ\).
\end{definition}

Для подсчета числа слабых композиций воспользуемся методом Stars\&Bars: пусть у
нас есть \(n\) единиц и \(k - 1\) перегородка между ними, значит всего получаем

\begin{align*}
  \Big|
    \text{Количество слабых композиций } n \text{ на } k \text{ частей}
  \Big|= \binom{n + k - 1}{k - 1}
\end{align*}

\begin{definition}
  Композицией числа \(n \ge 0\) на \(k\) частей называется решение уравнения
  \(b_{1} + \dotsc + b_{k} = n\) при условии \(b_{i} > 0, b_{i} \in \ZZ\).
\end{definition}

Количество композиций числа можно посчитать следующим образом: возьмем из \(n\)
\(k\) единиц (по одной для каждого слагаемого), а для оставшихся \(n - k\)
единиц решим задачу о слабой композиции. Итого получим

\begin{align*}
  \Big|
    \text{Количество композиций } n \text{ на } k \text{ частей}
  \Big| = \binom{n - 1}{k - 1}
\end{align*}

\begin{remark}
  Число композиций числа \(n\) на произвольное число частей
  (т.е. \(k = 1 \dots n\)) равно \(2^{n - 1}\).
\end{remark}
