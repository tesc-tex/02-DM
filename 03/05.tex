\subsection{%
  Преобразование НКА в ДКА.%
}

\underline{Алгоритм}:
\begin{enumerate}
  \item В начале алгоритма ДКА пустой. Будем постепенно по одной добавлять в
  него новые вершины.
  
  \item Изначально во множестве \(S\) содержится только одна стартовая вершина
  \(s\).

  \item Проходим по всем вершинам в \(S\).
  
  \item Проходим по всем символам алфавита.
  
  \item Для текущей вершины \(v\) и текущего символа \(c\) смотрим на множество
  вершин \(P\), в которое мы можем перейти.

  \item Создаем новую вершину \(p\), которая будет являться как бы объединением
  вершин в \(P\).

  \item Если такой вершины нет в ДКА, то добавляем её в ДКА и в множество
  \(S'\).

  \item В ДКА добавляем переход \(v \xrightarrow{c} p\).
  
  \item После того, как мы прошлись по всем вершинам в \(S\) и всем символам в 
  алфавите (шаги \(3-8\)) нужно обновить \(S\): \(S = S'\), а также очистить
  множество \(S'\).
\end{enumerate}

\todo Визуализация
