\subsection{%
  Мастер теорема.%
}

Мастер теорема позволяет давать асимптотическую оценку некоторым рекуррентным
соотношениям (которые чаще всего возникают в алгоритмах разделяй-и-властвуй).

\underline{Ограничения}:

Функция должна иметь вид

\begin{align*}
  T(n) = a \cdot T \left( \frac{n}{b} \right) + f(n) \\
  a \ge 1 \qquad b > 1 \qquad f(n) > 0
\end{align*}

Обозначим \(c_{crit} = \log_{b} a\). Тогда возможны три случая:

\begin{enumerate}
  \item \(
    f(n) \in O(n^{c}), c < c_{crit}
    \implies T(n) \in \Theta(n^{c_{crit}})
  \)

  \item \(f(n) \in \Theta(n^{c_{crit}} \log^{k} n)\). В зависимости от \(k\)
  есть три подслучая:

  \begin{enumerate}
    \item \(k > -1 \implies T(n) \in \Theta(n^{c_{crit}} \log^{k + 1} n)\)
    \item \(k = -1 \implies T(n) \in \Theta(n^{c_{crit}} \log \log n)\)
    \item \(k < -1 \implies T(n) \in \Theta(n^{c_{crit}})\)
  \end{enumerate}

  \item \(
    f(n) \in \Omega(n^{c}), c > c_{crit}
    \implies T(n) \in \Theta(f(n))
  \)
\end{enumerate}

\underline{Пример}:

\begin{align*}
  T(n) = 2 \cdot T(n / 2) + n \log n \\
  c_{crit} = \log_{2} 2 = 1 \\
  f(n) \in \Theta(n^{c_{crit}} \log^{1} n) \implies \text{2ой случай} \\
  k = 1 \implies \text{1ый подслучай} \\
  T(n) \in \Theta(n \log^{2} n)
\end{align*}
